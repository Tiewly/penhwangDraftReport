\chapter{\ifcpe บทนำ\else Introduction\fi}

\section{\ifcpe ที่มาของโครงงาน\else Project rationale\fi}
บริษัทต่าง ๆ ที่มีพนักงานและมีการจ่ายเงินเดือนตามเวลาเข้าออกงานของพนักงานก็จะมีวิธีการเช็คชื่อเข้า-ออกงานของพนักงานที่ต่างกัน 
หลากหลายรูปแบบ เช่น การเซ็นชื่อลงบนกระดาษ, การตอกบัตร หรือ การสแกนลายนิ้วมือ
แต่ในปัจจุบันมีบริษัทส่วนหนึ่งได้ตระหนักถึงปัญหาจากการใช้วิธีการเช็คชื่อเข้าออกงานแบบเดิม ๆ 
เช่น คำนวณเงินเดือนยากเพราะต้องทำการค้นหาข้อมูลจากเอกสารจำนวนมาก 
พนักงานทุจริตด้วยการตอกบัตรแทนกัน 
หรือ พนักงานตอกบัตรผิดใบ 
ซึ่งการใช้มนุษย์ในการบันทึกหรือจัดการข้อมูล มักจะเกิดความผิดพลาดที่เกิดจากมนุษย์(human error) 
และเกิดความล่าช้า โดยจะสังเกตได้จากในวันออกเงินเดือน พนักงานจะได้กลับบ้านช้ากว่าปกติเพราะต้องรอคำนวณเงินเดือนให้เสร็จเสียก่อน

จึงมีบริษัทส่วนหนึ่งเลือกที่จะใช้แอปพลิเคชันต่าง ๆ เพื่อที่จะจัดการแก้ไขปัญหาดังกล่าวข้างต้น 
เพราะสามารถเรียกดูข้อมูลได้ตลอดเวลา 
สรุปผลและคำนวณออกมาเป็นเงินเดือนได้อย่างรวดเร็ว 
สามารถป้องกันการทุจริตของพนักงาน 
รวมไปถึงจัดการการเดินเรื่องขอเอกสารให้มีความสะดวกรวดเร็วมากยิ่งขึ้น และ ลดปัญหาที่เกิดขึ้นจากการเก็บข้อมูลโดยใช้มนุษย์ไปพร้อมกัน

แต่แอปพลิเคชันเหล่านี้ก็ยังมีข้อเสีย เช่น 
พนักงานต้องทำการดาวน์โหลดแอปพลิเคชันไว้ในเครื่องส่วนตัวซึ่งพนักงานส่วนใหญ่ไม่เต็มใจที่จะดาวน์โหลดแอป โดยบางแอปพลิเคชันไม่ได้อำนวยความสะดวกในการใช้งานด้านต่าง ๆ เช่น
ไม่มีการแจ้งเตือนเมื่อพนักงานจะต้องเข้าทำงาน, กำลังจะเข้างานสาย, มีการเปลี่ยนแปลงเวลาการทำงานของตนเอง หรือ คำขอต่าง ๆ ของตนเองถูกยืนยัน/ปฎิเสธ
ซึ่งเป็นสิ่งที่แอปพลิเคชันควรจะรองรับ 
และการเช็คชื่อเข้าทำงานของพนักงานยังทำได้ช้าและมีหลายขั้นตอนทำให้เวลาที่บันทึกอยู่ในระบบและเวลาที่พนักงานเข้างานจริงต่างกันพอสมควร

ทางผู้พัฒนาเล็งเห็นปัญหาข้างต้นจึงได้พัฒนาโปรเจคนี้ขึ้นโดยการใช้ line chatbot มาพัฒนาต่อยอด
เพื่อให้สามารถทำงานตามที่แอปพลิเคชันเหล่านั้นทำได้
โดยรักษาข้อดีต่าง ๆ เอาไว้ และ แก้ปัญหาที่เกิดขึ้นจากการใช้แอปพลิเคชันเหล่านั้นด้วย

\section{\ifcpe วัตถุประสงค์ของโครงงาน\else Objectives\fi}
\begin{enumerate}
    \item พัฒนาแชทบอทที่สามารถทำงานหลัก ๆ ของแอปพลิเคชันเช็คชื่อพนักงานได้ คือ 
    \begin{enumerate}
    \item[1.1] จัดตารางเข้า-ออกให้กับพนักงานที่อยู่ในสายบังคับของตน
    \item[1.2] เช็คชื่อเข้า-ออกโดยใช้การถ่ายรูป พร้อมกับตรวจสอบที่อยู่ 
    \item[1.3] สามารถยื่นคำขอเปลี่ยนเวลาการทำงานของตนเอง
    \item[1.4] ตั้งค่าบริษัทเช่น การเพิ่ม-ลดพนักงาน และ สายบังคับของพนักงาน 
    \end{enumerate} 
    \item มีการแจ้งเตือนเมื่อพนักงานจะต้องเข้าทำงาน, กำลังจะเข้างานสาย, มีการเปลี่ยนแปลงเวลาการทำงานของตนเอง หรือ คำขอต่าง ๆ ของตนเองถูกยืนยัน/ปฎิเสธ
    \item ต้องเป็น chatbot ที่คุยด้วยแล้วเหมือนคุยกับคนจริง ๆ ที่กำลังหาข้อมูลมาตอบ/ทำเรื่องอยู่
\end{enumerate}

\section{\ifcpe ขอบเขตของโครงงาน\else Project scope\fi}

\subsection{\ifcpe ขอบเขตด้านฮาร์ดแวร์\else Hardware scope\fi}
\begin{itemize}
    \item Android version 4.4 เป็นต้นไป
\end{itemize}

\subsection{\ifcpe ขอบเขตด้านซอฟต์แวร์\else Software scope\fi}
\begin{itemize}
    \item ระบบบันทึกเวลาการเข้าออกงานของพนักงาน
    \item ระบบจัดการตารางเวลาทำงาน 
    \item ระบบแจ้งเตือนก่อนเวลาเข้างาน
    \item ระบบแจ้งเตือนเมื่อมีการเปลี่ยนแปลงตารางเวลาการทำงาน
    \item ระบบแจ้งเตือนเมื่อมีการอนุมัติ/ปฏิเสธคำของต่าง ๆ
    \item ระบบคำนวณเงินเดือน
\end{itemize}

\section{\ifcpe ประโยชน์ที่ได้รับ\else Expected outcomes\fi}
\begin{itemize}
    \item ช่วยอำนวยความสะดวกในการบันทึกเวลาเข้าออกงาน
    \item บันทึกข้อมูลไว้เพื่อง่ายต่อการตรวจสอบ
    \item ใช้เวลาในการจัดการเอกสารคำขอต่าง ๆ น้อยลง
    \item สามารถดูสิทธิ์การลาคงเหลือ และสรุปประวัติการบันทึกเวลาได้ง่ายและทันที
    \item หัวหน้างานสามารถติดตามการเข้าออกงานของลูกน้องได้แบบ Real-Time
    \item มีระบบการแจ้งเตือนทำให้สามารถจัดการหรือเตรียมตัวก่อนล่วงหน้า
\end{itemize}
\section{\ifcpe เทคโนโลยีและเครื่องมือที่ใช้\else Technology and tools\fi}
\begin{itemize}
    \item VS code
    \item Line application
    \item Line API Messenging
    \item Line Bot Designer​ \end{englang}
    \item Dialogflow
    \item firebase
    \item Nuxt.js
\end{itemize}
\section{\ifcpe แผนการดำเนินงาน\else Project plan\fi}
\begin{plan}{7}{2020}{3}{2021}
    \planitem{7}{2020}{9}{2020}{ศึกษาแอปฯปัจจุบัน}
    \planitem{8}{2020}{9}{2020}{ศึกษาการสร้างแชทบอท​}
    \planitem{7}{2020}{9}{2020}{ออกแบบระบบ}
    \planitem{10}{2020}{10}{2020}{ออกแบบแชทบอท​}
    \planitem{7}{2020}{10}{2020}{ออกแบบเว็บ​}
    \planitem{7}{2020}{10}{2020}{ออกแบบฐานข้อมูล}
    \planitem{11}{2020}{12}{2020}{สร้างแชทบอท}
    \planitem{11}{2020}{1}{2021}{เชื่อมแชทบอทกับฐานข้อมูล}
    \planitem{11}{2020}{1}{2021}{สร้างเว็บ}
    \planitem{11}{2020}{2}{2021}{เชื่อมเว็บกับแชทบอท}
    \planitem{11}{2020}{3}{2021}{ทดสอบและแก้ไข Bugs}
\end{plan}

\section{\ifcpe บทบาทและความรับผิดชอบ\else Roles and responsibilities\fi}
\begin{itemize}
    \item น.ส.ธนันพร ยานะ
    \begin{itemize}
        \item Web designer​ \end{englang}
        \item Chatbot designer​​ \end{englang}
        \item Frontend developer​ \end{englang}
    \end{itemize}
    \item นายศรัณญ์ ซือสุวรรณ
    \begin{itemize} 
        \item Database admin
        \item Backend developer​​ \end{englang}
    \end{itemize}
\end{itemize}

\section{\ifcpe%
ผลกระทบด้านสังคม สุขภาพ ความปลอดภัย กฎหมาย และวัฒนธรรม
\else%
Impacts of this project on society, health, safety, legal, and cultural issues
\fi}

แนวทางและประโยชน์ในการประยุกต์ใช้งานโครงงานกับงานในด้านอื่นๆ รวมถึงผลกระทบในด้านสังคมและสิ่งแวดล้อมจากการใช้ความรู้ทางวิศวกรรมที่ได้
