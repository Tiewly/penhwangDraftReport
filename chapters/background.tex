\chapter{\ifcpe ทฤษฎีที่เกี่ยวข้อง\else Background Knowledge and Theory\fi}

\quad การทำโครงงาน เริ่มต้นด้วยการศึกษาค้นคว้า ทฤษฎีที่เกี่ยวข้อง หรือ งานวิจัย/โครงงาน ที่เคยมีผู้นําเสนอไว้แล้วซึ่งเนื้อหาในบทนี้ก็จะเกี่ยวกับการอธิบายถึงสิ่งที่เกี่ยวข้องกับโครงงาน เพื่อให้ผู้อ่านเข้าใจเนื้อหาในบทถัด ๆ ไปได้ง่ายขึ้น เนื้อหาในบทนี้จะแบ่งออกเป็นสี่ส่วนหลัก ๆ คือ ส่วนที่เป็นระบบบันทึกเวลาเข้า-ออกงานผ่านระบบออนไลน์, ระบบการหาตำแหน่งทั่วโลกหรือ GPS, ส่วน Line และ ส่วนการเก็บข้อมูลและประมวลผล ดังนี้ 

\section{ระบบบันทึกเวลาเข้า-ออกงานผ่านระบบออนไลน์}
\quad ระบบบันทึกเวลาเข้า-ออกงานผ่านระบบออนไลน์ (Online Time Attendance Management) คือ ระบบลงเวลาเข้า-ออกงานโดยผ่านอุปกรณ์ดิจิตอลต่าง ๆ ตั้งแต่เครื่องสแกนลายนิ้วมือ (Finger Scanner), โทรศัพท์มือถือ (Smart Phone), Tablet ตลอดจนอุปกรณ์ต่าง ๆ ที่สามารถยืนยันตัวตนผู้ใช้งานได้อย่างชัดเจน ปลอดภัย และเชื่อมต่อสู่ระบบข้อมูลกลางอย่างระบบ Cloud ได้ ซึ่งการยืนยันตัวตนในรูปแบบนี้สามารถเข้าระบบที่ไหนก็ได้ในโลกเป็นการบันทึกเวลาทำงานได้แบบ Real time ที่เชื่อมต่อข้อมูลสู่ฐานข้อมูลกลางเดียวกัน และยังสามารถระบุตำแหน่งและเวลาได้ชัดเจนสามารถเป็นหลักฐานประกอบได้อย่างมีประสิทธิภาพ ซึ่งจะช่วยลดปัญหาการเก็บข้อมูลโดยระบบสแกนนิ้วมือหรือระบบตอกบัตรแบบเดิม ที่ไม่เชื่อมต่อสู่ระบบออนไลน์ได้เป็นอย่างดี ทั้งยังสามารถประมวลผลข้อมูลได้หลากหลายรูปแบบ แม่นยํา ชัดเจน และรวดเร็วอีกด้วย

\subsection{ความสำคัญของการบันทึกเวลา}
\subsubsection{บันทึกหลักฐานการทำงาน}
\quad การที่ชั่วโมงในการทำงานนั้นจะมีส่วนเกี่ยวช้องในการคำนวนเงินเดือน, หรือหักเงินจ้างในกรณีที่ทำงานไม่ครบตามชั่วโมงที่กำหนด 
หากไม่มีหลักฐานชัดเจนกันทั้งสองฝ่ายก็อาจเกิดการถกเถียงกันได้ภายหลัง หรือไม่มีข้อยุติที่ชัดเจน บริษัทจึงต้องมีหลักฐานชั่วโมงการทำงานของพนักงาน และพนักงานก็ต้องมีหลักฐานในการยืนยันตนเองด้วยเช่นกัน
หลักฐานนี้ยังมีประโยชน์ในอีกหลากหลายด้านรวมถึงส่วนของนอกองค์กรด้วย เช่น 
หลักฐานพนักงานที่ส่งให้หน่วยงานรัฐ, หลักฐานการทำงานที่ใช้รับรองกับการทำธุรกรรม, หรือการบันทึกชั่วโมงการทำงานสำหรับบางสาขาอาชีพที่จำเป็นต้องใช้หลักฐานด้านนี้ เป็นต้น 
\subsubsection{ควบคุมการทำงานได้ง่าย}
\quad ระบบลงเวลาหรือบันทึกเวลาเข้า-ออกการทำงานจะสามารถทำให้ควบคุมการทำงานของพนักงานได้ง่าย เพราะไม่จำเป็นต้องยจัดการกับเอกสารต่าง ๆ และพนักงานแต่ละคนทำงานอย่างอิสระต่อกัน 
\subsubsection{ใช้บริหารงบประมาณ}
\quad หลักฐานในด้านจำนวนชั่วโมงในการทำงานเช่น การขาด ลา หรือมาสาย ล้วนแล้วแต่เป็นข้อมูลสำคัญสำหรับการคำนวนรายได้ของพนักงานหรือรายจ่ายของบริษัท 
การเก็บหลักฐานให้ชัดเจนจะทำให้สามารถบริหารงบประมาณ ตลอดจนบริหารการใช้จ่ายงบขององค์กรได้อย่างมีประสิทธิภาพ ทั้งยังสร้างความเป็นธรรมทั้งสองฝ่าย

\subsection{ประโยชน์ของการบันทึกเวลาเข้า-ออกงานผ่านระบบออนไลน์}
\subsubsection{บริหารจัดการข้อมูลได้รวดเร็ว ว่องไว สะดวกสบาย}
\quad การบันทึกเวลาการทำงานในรูปแบบเดิมอย่าง เช่นการตอกบัตรนั้นใช้เวลาจำนวนมาก และยังต้องใช้เวลาอีกมากในการนำเอาข้อมูลไปใช้ประโยชน์ 
การบันทึกเวลาการทำงานในรูปแบบใหม่นี้จะทำให้บริหารจัดการข้อมูลเป็นไปแบบอัตโนมัติ โดยเฉพาะการคำนวณต่าง ๆ ตั้งแต่การคำนวนชั่วโมงการทำงาน การคำนวนชั่วโมงที่ไม่ได้ทำงาน การคำนวนอัตราเงินเดือนของพนักงานแต่ละคน ไปจนถึงการวิเคราะห์ข้อมูลต่าง ๆ นั้น ทำได้อย่างรวดเร็วผ่านโปรแกรม สามารถทราบข้อมูลได้แบบ Real time และสามารถประมวลข้อมูลได้หลากหลายลักษณะตามต้องการอย่างทันท่วงที 
\subsubsection{ทุกอย่างรวมอยู่ในที่เดียวและเบ็ดเสร็จในจุดเดียว (All in One \& One Stop Service)}
\quad การเข้า-ออกงานแบบบันทึกเวลาระบบดั้งเดิมอย่างการตอกบัตร หรือสแกนนิ้วมือนั้นเป็นเพียงแค่การบันทึกเวลาเข้า-ออกเท่านั้น 
การขาด ลา มาสาย หรือการดำเนินการเรื่องชั่วโมงการทำงานอื่น ๆ นั้นยังคงเป็นระบบที่ใช้มนุษย์จัดการบันทึก 
แต่สำหรับการบันทึกเวลาเข้า-ออกงานผ่านแอปพลิเคชันบริหารจัดการบุคคล (Attendance Management Application) 
นั้นสามารถทำทุกอย่างได้ในที่เดียวทั้งในส่วนของพนักงานเองและฝ่ายทรัพยากรบุคคล ทำผ่านระบบฐานข้อมูลกลางที่ผ่านระบบ Cloud ซึ่งเป็นข้อมูลหนึ่งเดียวกัน ไม่ซับซ้อน ไม่ต้องมีข้อมูลหลายแหล่ง และ
เป็นบริการแบบ One Stop Service ซึ่งมีความหมายว่าแอปพลิเคชันเดียวสามารถจัดการได้ทั้งระบบ 
\subsubsection{สะดวกทุกที่ ทุกเวลา (Anywhere \& Anytime)}
\quad พนักงานตลอดจนฝ่ายทรัพยากรบุคคล (HR) เองสามารถเข้าแอปพลิเคชันได้ทุกที่ทุกเวลาในการบันทึกเวลาเข้างานตามจริง และเหมาะสม
ทั้งข้อมูลยังถูกต้องชัดเจนด้วย สามารถเช็คอินได้ทุกแห่งทั่วโลก ทุกเวลา แม้เวลาต่าง Time Zone กัน ซึ่งบางครั้งต้องไปทำงานยังต่างประเทศ
ก็สามารถบันทึกเวลาทำงานตามจริงได้ 
\subsubsection{หลักฐานที่ชัดเจนแน่นอน มีข้อมูลอ้างอิงที่น่าเชื่อถือ (Working Hours Evidence)}
\quad การบันทึกเวลาเข้าออกงานผ่านแอปพลิเคชันบริหารจัดการบุคคล (Attendance Management Application) 
นั้นยังทำให้ข้อมูลชัดเจน แน่นอน ถูกต้อง ไม่ได้อ้างอิงจากคำบอกเล่าของพนักงาน เพราะแอปพลิเคชันจะทำงานร่วมกับเทคโนโลยีระบบระบุตำแหน่งและเวลาได้ด้วย
\subsubsection{พนักงานสามารถบริหารจัดการวันลาได้ด้วยตัวเอง}
\quad ตามกฎหมายแล้วพนักงานทุกคนมีสิทธิ์ในการลางาน เมื่อใช้แอปพลิเคชันจะทำให้ทุก ๆ คนสามารถรู้สิทธิ์ที่ใช้ไป ตลอดจนสิทธิ์ที่เหลือ 
รวมถึงจัดการการลาต่าง ๆ ได้ด้วยตนเอง ไม่จำเป็นต้องรบกวนฝ่ายบุคคล เพราะทุกอย่างจะถูกคำนวนและปรากฎในแอปพลิเคชันอย่างอัตโนมัติ 
ทำให้ง่ายต่อการบริหารวันลาของตนด้วยตัวเอง โดยไม่ต้องไปรบกวนพนักงานคนอื่นหรือฝ่ายบุคคลจนเกินไป 
\subsubsection{ไม่สร้างความขัดแย้งให้กับบุคลากรในองค์กร}  
\quad ระบบการบันทึกเวลาแบบเดิมเป็นเพียงการบันทึกเวลาเท่านั้น การขาด ลา มาสาย ยังคงต้องมีการติดต่อสอบถามข้อมูลระหว่างพนักงาน 
ซึ่งอาจมีการสื่อสารระหว่างกันด้วยอารมณ์ ทำให้ในบางครั้งอาจทำให้เกิดการทะเลาะเบาะแว้ง จนถึงสร้างความบาดหมางหรือความแตกแยกให้กับบุคลากรในองค์กรได้ด้วย 
การใช้แอปพลิเคชันบันทึกเวลาสามารถลดความขัดแย้งในส่วนนี้ไปได้ เพราะแอปพลิเคชันจะเข้ามาเป็นตัวกลางระหว่างพนักงาน คอยทำหน้าที่สอบถามข้อมูลที่จำเป็นอย่างอัตโนมัติ
\subsubsection{ประหยัดทรัพยากร และ งบประมาณ}
\quad การใช้เทคโนโลยีเข้ามาช่วยบันทึกเวลาเข้าออกงานผ่านแอปพลิเคชันบริหารจัดการบุคคล (Attendance Management Application) 
นั้นจะช่วยทำให้องค์กรประหยัดทรัพยากรหลายอย่าง เช่นกระดาษในการบันทึกเอกสารต่าง ๆ เครื่องมือตอกบัตร เครื่องมือสแกนลายนิ้วมือ เพราะแอปพลิเคชันไม่ต้องการทรัพยากรเหล่านั้น 
และการประหยัดทรัพยากรนั้นส่งผลต่อการประหยัดงบประมาณโดยตรง เพราะไม่ต้องมีงบจัดซื้ออุปกรณ์ หรืองบในการบำรุงรักษาซ่อมแซม ใช้เพียงแค่งบในการซื้อเทคโนโลยีเท่านั้น 
กรณีนี้องค์กรที่ใหญ่มักจะเห็นผลในการประหยัดงบประมาณได้ชัดเจนกว่า 
\subsubsection{ลดใช้บุคลากร}  
\quad เมื่อมีการใช้เทคโนโลยี จะทำให้ลดอัตราจ้างพนักงานลงได้ เพราะเมื่อทุกคนร่วมกันใช้แอปพลิเคชัน 
จะทำให้การทำงานสะดวกสบายขึ้น การจัดการง่ายขึ้น และไม่จำเป็นต้องใช้คนจัดการมาก ทำให้สามารถลดปริมาณคนในฝ่ายทรัพยากรมนุษย์ได้ 
โดยที่องค์กรสามารถบริหารงานและเงินได้อย่างมีประสิทธิภาพเหมือนเดิม 
\subsubsection{สามารถอัพเดท (Update) เทคโนโลยีใหม่ได้เสมอ}
\quad การอัพเดทเทคโนโลยีตลอดจนสิ่งที่เป็นประโยชน์สมัยใหม่กับซอฟท์แวร์ต่าง ๆ นั้นทำได้ง่าย สะดวก รวดเร็ว และประหยัดกว่าการอัพเดทฮาร์ดแวร์ 
โดยเฉพาะเทคโนโลยีดิจิตอลที่มีการ Update ได้แบบ Real time จะทำให้ระบบสามารถมีอะไรใหม่ ๆ มาเป็นตัวช่วยเพิ่มเติมได้เสมอ 
ประหยัดกว่าการใช้ฮาร์ดแวร์หรือซอฟท์แวร์ระบบปิด 
\subsubsection{พักงานเกิดความสบายใจ สุขภาพจิตดี เพิ่มประสิทธิภาพในการทำงาน}
\quad การเซ็นต์ชื่อ การใช้เครื่องตอกบัตร หรือการสแกนลายนิ้วมือนั้นอาจทำให้พนักงานเกิดอารมณ์หงุดหงิด และรู้สึกสูญเสียอิสระของตนเอง 
การใช้แอปพลิเคชันบันทึกเวลาที่ออฟฟิศก็จะทำให้พนักงานรู้สึกเป็นอิสระมากขึ้น เพราะไม่จำเป็นต้องต่อแถวรอ หรือเป็นห่วงว่าเครื่องจะทำงานผิดพลาด
และยังสามารถทำให้พนักงานทำงานได้มีประสิทธิภาพขึ้น 
\subsubsection{บริษัทยังคงสร้างวินัยให้กับการทำงานขององค์กรได้เช่นเดิม}  
\quad การบันทึกเวลาเข้าออกงานผ่านแอปพลิเคชันบริหารจัดการบุคคล (Attendance Management Application) 
ในระบบใหม่นี้ยังคงมีการบันทึกข้อมูลที่องค์กรจำเป็นต้องใช้เหมือนเดิม และยังคงสร้างวินัยให้กับพนักงานได้เช่นเดิม 
เพียงแต่ว่าจะเป็นวินัยในการบริการจัดการเวลาแบบใหม่ที่ไม่จำเป็นจะต้องอยู่กับที่เสมอไป แต่ฝึกความรับผิดชอบในการทำงานและการใช้เวลาให้คุ้มค่าได้ดี 
มีวินัยในการบันทึกข้อมูล วินัยในการทำงานที่ชัดเจน และมีหลักฐานในการทำงานที่ชัดเจน เป็นข้อมูลอ้างอิงที่น่าเชื่อถือขึ้นได้ด้วย 
\cite{hrnote}
\section{ระบบการหาตำแหน่งทั่วโลก หรือ GPS}
\quad ระบบการหาตำแหน่งทั่วโลก หรือ GPS (Global Positioning System) คือ ระบบการนำทางด้วยดาวเทียมซึ่งประกอบด้วยดาวเทียมอย่างน้อย 24 ดวง GPS สามารถปฏิบัติการได้ในทุกสภาพอากาศ ทุกที่ในโลก ตลอด 24 ชั่วโมงต่อวัน และไม่มีค่าลงทะเบียนหรือค่าธรรมเนียมในการตั้งค่า กระทรวงกลาโหมสหรัฐ (USDOD) แต่เดิมปล่อยดาวเทียมให้โคจรสำหรับการปฏิบัติงานทางทหาร แต่ในทศวรรษ 1980 เป็นต้นมาก็เริ่มกำหนดให้พลเรือนสามารถเข้าถึงการใช้งานดาวเทียมได้ โดยระบบ GPS ประกอบไปด้วย 3 ส่วนหลัก คือ  
\begin{itemize}
  \item[1] ส่วนอวกาศ ประกอบด้วยเครือข่ายดาวเทียมหลัก 3 ค่าย คือ อเมริกา รัสเซีย ยุโรป โดยของอเมริกา ชื่อ NAVSTAR (Navigation Satellite Timing and Ranging GPS) มีดาวเทียม 28 ดวง ใช้งานจริง 24 ดวง อีก 4 ดวงเป็นตัวสำรอง บริหารงานโดย Department of Defenses มีรัศมีวงโคจรจากพื้นโลก 20,162.81 กม.หรือ 12,600 ไมล์ ดาวเทียมแต่ละดวงใช้ เวลาในการโคจรรอบโลก 12 ชั่วโมง ของยุโรป ชื่อ Galileo มี 27 ดวง บริหารงานโดย ESA หรือ European Satellite Agency จะพร้อมใช้งานในปี 2008 ของรัสเซีย ชื่อ GLONASS หรือ Global Navigation Satellite บริหารโดย Russia VKS (Russia Military Space Force) ในขณะนี้ภาคประชาชนทั่วโลกสามารถใช้ข้อมูลจากดาวเทียมของทางอเมริกา (NAVSTAR) ได้ฟรี เนื่องจาก นโยบายสิทธิการเข้าถึงข้อมูลและข่าวสารสำหรับประชาชนของรัฐบาลสหรัฐ จึงเปิดให้ประชาชนทั่วไปสามารถใช้ข้อมูลดังกล่าวในระดับความแม่นยำที่ไม่เป็นภัยต่อความมันคงของรัฐ  
  \item[2] ส่วนควบคุม ประกอบด้วยสถานีภาคพื้นดิน สถานีใหญ่อยู่ที่ Falcon Air Force Base ประเทศ อเมริกา และศูนย์ควบคุมย่อยอีก 5 จุด กระจายไปยังภูมิภาคต่าง ๆ ทั่วโลก
  \item[3] ส่วนผู้ใช้งาน ผู้ใช้งานต้องมีเครื่องรับสัญญาณที่สามารถรับคลื่นและแปรรหัสจากดาวเทียมเพื่อนำมาประมวลผลให้เหมาะสมกับการใช้งานในรูปแบบต่าง ๆ 
\end{itemize}
\subsection{การทำงานของ GPS}
\quad ดาวเทียม GPS ประกอบด้วยดาวเทียม 24 ดวง โดยแบ่งเป็น 6 รอบวงโคจร การโคจรจะเอียงทำมุมเอียง 55 องศากับเส้นศูนย์สูตร (Equator) ในลักษณะสานกันคล้ายลูกตะกร้อ แต่ละวงโคจรมีดาวเทียม 4 ดวง รัศมีวงโคจรจากพื้นโลก 20,162.81 กม. หรือ 12,600 ไมล์ ดาวเทียมแต่ละดวงใช้ เวลาในการโคจรรอบโลก 12 ชั่วโมง 
\quad GPS ทำงานโดยการรับสัญญาณจากดาวเทียมแต่ละดวง โดยสัญญาณดาวเทียมนี้ประกอบไปด้วยข้อมูลที่ระบุตำแหน่งและเวลาขณะส่งสัญญาณ ตัวเครื่องรับสัญญาณ GPS จะต้องประมวลผลความแตกต่างของเวลาในการรับสัญญาณเทียบกับเวลาจริง ณ ปัจจุบันเพื่อแปรเป็นระยะทางระหว่างเครื่องรับสัญญาณกับดาวเทียมแต่ละดวง ซึ่งได้ระบุมีตำแหน่งของมันมากับสัญญาณดังกล่าวข้างต้น เพื่อให้เกิดความแม่นยำในการค้นหาตำแหน่งด้วยดาวเทียม ต้องมีดาวเทียมอย่างน้อย 4 ดวง เพื่อบอกตำแหน่งบนผิวโลก ซึ่งระยะห่างจากดาวเทียมทั้ง 3 กับเครื่อง GPS จะสามารถระบุตำแหน่งบนผิวโลกได้ หากพื้นโลกอยู่ในแนวระนาบแต่ในความเป็นจริงพื้นโลกมีความโค้งเนื่องจากสัณฐานของโลกมีลักษณะกลม ดังนั้นดาวเทียมดวงที่ 4 จะทำให้สามารถคำนวณเรื่องความสูงเพื่อทำให้ได้ตำแหน่งที่ถูกต้องมากขึ้น นอกจากนี้ความแม่นยำของการระบุตำแหน่งนั้นขึ้นอยู่กับตำแหน่งของดาวเทียมแต่ละดวง กล่าวคือถ้าระยะห่างระหว่างดาวเทียมที่ใช้งานอยู่ห่างกันย่อมให้ค่าที่แม่นยำกว่าที่อยู่ใกล้กัน และยิ่งมีจำนวนดาวเทียมที่รับสัญญาณได้มากก็ยิ่งให้ความแม่นยำมากขึ้น ความแปรปรวนของชั้นบรรยากาศชั้นบรรยากาศประกอบด้วยประจุไฟฟ้า ความชื้น อุณหภูมิ และความหนาแน่นที่แปรปรวนตลอดเวลา คลื่นเมื่อตกกระทบ กับวัตถุต่าง ๆ จะเกิดการหักเหทำให้สัญญาณที่ได้อ่อนลง และสิ่งแวดล้อมในบริเวณรับสัญญาณเช่นมีการบดบังจากกระจก ละอองน้ำ ใบไม้ จะมีผลต่อค่าความถูกต้องของความแม่นยำ เนื่องจากถ้าสัญญาณจากดาวเทียมมีการหักเหก็จะทำให้ค่าที่คำนวณได้จากเครื่องรับสัญญาณเพี้ยนไป และสุดท้ายก็คือประสิทธิภาพของเครื่องรับสัญญาณว่ามีความไวในการรับสัญญาณแค่ไหนและความเร็วในการประมวณผลด้วย 
\quad การวัดระยะห่างระหว่างดาวเทียมกับเครื่องรับทำได้โดยใช้สูตรคำนวณ ระยะทาง = ความเร็ว * ระยะเวลา วัดระยะเวลาที่คลื่นวิทยุส่งจากดาวเทียมมายังเครื่องรับ GPS คูณด้วยความเร็วของคลื่นวิทยุจะเท่ากับระยะทางที่เครื่องรับ อยู่ห่างจากดาวเทียม โดยเวลาที่วัดได้มาจากนาฬิกาของดาวเทียมที่มีความแม่นยำสูงมีความละเอียดถึงนาโนวินาที และมีการสอบทวนเสมอๆกับสถานีภาคพื้นดิน องค์ประกอบสุดท้ายก็คือตำแหน่งของดาวเทียมแต่ละดวงในขณะที่ส่งสัญญาณมาว่าอยู่ที่ใด (Almanac) มายังเครื่องรับ GPS โดยวงโคจรของดาวเทียมได้ถูกกำหนดไว้ล่วงหน้าแล้วเมื่อถูกส่งขึ้นสู่อวกาศ สถานีควบคุมจะคอยตรวจสอบการโคจรของดาวเทียมอยู่ตลอดเวลาเพื่อทวนสอบความถูกต้อง 
\quad ในการคำนวณตำแหน่ง 2 มิติของคุณ (ละติจูดและลองจิจูด) และติดตามการเคลื่อนที่ ตัวรับสัญญาณ GPS ต้องถูกล็อกเข้ากับสัญญาณของดาวเทียมอย่างน้อย 3 ดวง และด้วยดาวเทียม 4 ดวงขึ้นไป ตัวรับสัญญาณจะสามารถระบุตำแหน่ง 3 มิติของคุณ (ละติจูด ลองจิจูด และระดับความสูง) โดยทั่วไปแล้ว ตัวรับสัญญาณ GPS จะติดตามดาวเทียม 8 ดวงขึ้นไป แต่นั่นก็ขึ้นอยู่กับเวลาในแต่ละวันและสถานที่บนโลกที่คุณอยู่ 
\cite{gps} \cite{gps2}
\subsection{ประโยชน์ของการใช้การลงทะเบียนแบบ GPS}
\subsubsection{ประโยชน์ในแง่มุมของธุรกิจ}
\quad ประโยชน์แรกที่บริษัทจะได้รับคือ ชั่วโมงแรงงานลดลง จนถึงตอนนี้บริษัทจำเป็นต้องตรวจสอบข้อมูลจากการใช้บัตรเวลาหรือบัตรตอก เพื่อการคํานวณเงินเดือน แต่ในตอนนี้สามารถทำโปรแกรมจัดการเหล่านี้ได้ด้วยการใช้ระบบการจัดการเวลาและระบบเงินเดือนได้ ดังนั้น ชั่วโมงแรงงานของการทำบัญชีเงินเดือนจะลดลง ไม่เพียงแต่ฝ่ายทรัพยากรบุคคลเท่านั้น แต่ส่งไปถึงฝ่ายบัญชีที่ทำให้มีประสิทธิผลมากขึ้นและเป็นการป้องกันไม่ให้เกิดความผิดพลาดได้ เช่นในกรณีที่ใช้บัตรเวลาอาจจะง่ายต่อการแก้ไขข้อมูลเวลาเข้าทำงาน ซึ่งฝ่าย HR อาจจะไม่รู้ว่ามีการแก้ไขข้อมูล ดังนั้นข้อมูลที่ผิดจึงเกิดขึ้นได้ แต่ในกรณีที่มีการลงทะเบียนแบบ GPS ข้อมูลตำแหน่งจะถูกบันทึกไว้เมื่อคุณลงทะเบียน ดังนั้นจึงเป็นการป้องกันข้อมูลผิดพลาดของเวลาเข้าหรือออกงาน 
\subsubsection{ประโยชน์ในแง่มุมของพนักงาน}
\quad ประโยชน์สำหรับพนักงานคือการลดเวลายุติธรรม ในกรณีใช้บัตรเวลาหรือบัตรตอก อาจจะเสียเวลาตรงที่คุณต้องเข้ามาตอกบัตรที่บริษัทก่อน ทั้ง ๆ ที่วันนั้นคุณต้องออกไปทำงานข้างนอก ในทางกลับกัน การลงทะเบียนแบบ GPS จะไม่ถูกผูกไว้กับสถานที่ใด ๆ จึงสามารถปรับประสิทธิภาพในการทำงานได้  และตามที่กล่าวมาข้างต้นเกี่ยวกับประโยชน์สำหรับบริษัท ซึ่งสามารถป้องกันข้อมูลที่ผิดพลาดได้ เพราะการลงทะเบียนแบบ GPS สามารถจับเวลาและสถานการณ์ทำงานได้อย่างถูกต้อง 
\cite{hrnotegps}
\section{LINE application}
\quad LINE คือ แอปพลิเคชันที่ผสมผสานบริการ Messaging และ Voice Over IP นำมาผนวกเข้าด้วยกัน จึงทำให้เกิดเป็นแอปพลิชันที่สามารถแชท สร้างกลุ่ม ส่งข้อความ โพสต์รูปต่าง ๆ  หรือจะโทรคุยกันแบบเสียงก็ได้  โดยข้อมูลทั้งหมดไม่ต้องเสียเงิน หากเราใช้งานโทรศัพท์ที่มีแพคเกจอินเทอร์เน็ตอยู่แล้ว แถมยังสามารถใช้งานร่วมกันระหว่าง iOS และ Android รวมทั้งระบบปฏิบัติการอื่น ๆ ได้อีกด้วย การทำงานของ LINE นั้น มีลักษณะคล้าย ๆ กับ WhatsApp ที่ต้องใช้เบอร์โทรศัพท์เพื่อยืนยันการใช้งาน แต่ LINE ได้เพิ่มลูกเล่นอื่น ๆ เข้ามา ทำให้ LINE มีจุดเด่นที่เหนือกว่า WhatsApp  

\subsection{จุดเด่นของ LINE} 
\subsubsection{การสนทนาด้วยเสียงฟรี (Free voice calls)} 
\quad ผู้ใช้งานสามารถโทรหาผู้ที่ใช้ LINE ด้วยกันได้ โดยใช้งานผ่านเครือข่าย 3G และ Wi-Fi เพื่อส่งข้อมูลรูปแบบเสียง โดยไม่มีค่าใช้จ่ายใด ๆ  
\subsubsection{ส่งข้อความแบบวิดีโอและเสียง (Send videos \& voice message)} 
\quad นอกจากการแชทด้วยการส่งข้อความแบบปกติแล้ว LINE ยังสามารถอัดภาพวิดีโอหรือเสียงแล้วส่งไปให้เพื่อน ๆ ได้อีกด้วย โดยสามารถส่งได้เป็นคลิปวิดีโอหรือเสียงในแบบสั้น ๆ ความยาวไม่กี่วินาที   
\subsubsection{สติกเกอร์การ์ตูน (Stickers \& Emoticons)}
\quad อีกหนึ่งความสนุกของแชททั่วไปที่ขาดไม่ได้ก็คืออีโมติคอนน่ารัก ๆ ที่ช่วยเพิ่มสีสันให้การแชทสนุกสนานยิ่งขึ้น และสำหรับ LINE มีทั้ง Stickers และ Emoticons รูปแบบต่าง ๆ และยังเลือกดาวน์โหลดเพิ่มเติมได้อีกด้วย ทำให้ผู้ใช้งานหลายคนติดอกติดใจกับ Stickers และ Emoticons น่ารัก ๆ ของ LINE  
\subsubsection{ปรับแต่งภาพพื้นหลัง (Customizable Wallpaper)}
\quad สามารถเปลี่ยน Wallpaper ในหน้าต่างแชทได้ โดยจะมีภาพ Wallpaper มาให้ทั้งหมด 23 แบบ และสามารถเพิ่ม Wallpaper ที่ต้องการ โดยนำรูปที่อยู่ในโทรศัพท์มือถือมาใช้งานเป็น Wallpaper ได้  
\subsubsection{การสนทนาแบบกลุ่ม (Group chat)} 
\quad LINE สามารถสร้างกลุ่มเพื่อพูดคุยกันได้ หากต้องการความเป็นส่วนตัว อยากคุยเฉพาะกลุ่ม LINE ก็สามารถสร้างกลุ่มเอาไว้พูดคุยได้ 
\subsubsection{Timeline} 
\quad LINE มีความเป็นโซเชียลเน็ตเวิร์กในตัว มี Timeline ให้สามารถอัพเดทสถานะ โพสต์รูป คอมเมนต์ หรือกดถูกใจได้เหมือนกับ Facebook เลยทีเดียว 
\subsubsection{การเพิ่มเพื่อน (Add friends / Contacts)}
\quad LINE สามารถเพิ่ม Contacts จากรายชื่อในโทรศัพท์หากมีเพื่อนคนไหนใช้แอปพลิเคชันนี้อยู่ จะมีสัญลักษณ์ LINE แสดงให้เห็นและสามารถเพิ่มเป็นเพื่อนได้ทันที QR Code สามารถสแกน QR Code ของเพื่อนเพื่อเพิ่มเป็นเพื่อนใน LINE และสามารถสร้าง QR Code ของเราเอง เพื่อใช้สำหรับให้เพื่อน ๆ คนอื่น มาสแกน QR Code เพื่อเพิ่มเพื่อนใน LINE ได้ Shake it! เขย่าโทรศัพท์มือถือ เป็นวิธีการเพิ่มเพื่อนที่เจ๋งสุด ๆ ของ LINE ใช้ในกรณีที่ทั้งสองโทรศัพท์สองเครื่องอยู่ด้วยกัน เมื่อเขย่าเครื่องพร้อม ๆ กัน ก็สามารถเพิ่มเป็นเพื่อนกันได้ Search by ID คือ เราสามารถค้นหาเพื่อนได้จาก ID (คล้าย ๆ กับ PIN ของ BB) โดยการพิมพ์ ID ของเพื่อนที่ต้องการ 

\subsection{LINE Messaging API} 
\quad Line Messaging API คือ การสื่อสารระหว่างบริการของคุณและผู้ใช้ LINE เป็นการสื่อสารแบบสองฝ่าย จะทำให้คุณสามารถให้บริการได้ในห้องแชท LINE เพื่อการให้บริการที่เหมาะสมสำหรับผู้ใช้ LINE แต่ละคนและ Messaging API จะส่งและรับข้อมูลระหว่าง Server ของคุณและแอพ LINE ผ่านทาง Server ของทาง LINE การส่งคำขอจะใช้ API แบบ JSON Messaging API ทำการเชื่อมต่อระหว่าง User ผ่านทาง LINE official account ซึ่ง Messaging API จะสามารถตอบรับเพื่อนรวมถึงส่งข้อความหา User คนอื่น ๆ ที่ Add account เราเป็นเพื่อนโดยผ่านหน้า LINE Manager ที่เราตั้งไว้หรือส่งออกจากจาก Server ของเราก็ได้ในรูปแบบ interactive โต้ตอบ  การใช้งาน Messaging API ทำให้คุณสามารถส่งข้อมูลระหว่าง Server ของเรา ไปยัง User LINE ผ่านทาง LINE Platform ซึ่ง Request ที่ใช้ส่งข้อมูลต้องอยู่ในรูป JSON format โดยตัว Server เราจะต้องเชื่อมต่อกับ LINE Platform และเมื่อ มี User เพิ่ม Account LINE เราเป็นเพื่อน หรือ ส่งข่อความมาหาเรา ทาง LINE Platform จะทำการส่ง Request มายัง Server ที่เราลงทะเบียนผูกไว้กับ LINE account นั้นทันที วิธีนี้เรียกว่า Webhook ซึ่งมันทำให้ผู้ใช้งานรู้สึกเหมือนกับว่าได้โต้ตอบกับคนจริง ๆ 
\cite{lineAPI}
\subsection{LINE Bot} 
\quad LINE Bot คือ Line Official Account ที่ได้นำ Messaging API มาใช้ เป็นบริการ API ตัวหนึ่งที่เปิดให้บริการสำหรับนักพัฒนา โดยเจ้าของ Line Official Account จะทำการกำหนดหรือตั้งค่าไว้ด้านหลังบ้านของบริการ เพื่อให้สามารถโต้ตอบกับผู้ใช้งานได้โดยที่ไม่ต้องใช้คนมาเป็นคนตอบ ซึ่งนี่คือข้อดีของการใช้บริการตอนนี้ เพราะนอกจากจะทำให้ผู้ใช้ใช้งานได้ง่ายมากขึ้นแล้ว ผู้ที่เป็นแอดมินก็จะสะดวกสบายมากขึ้นเช่นกัน เพราะไม่ต้องมาคอยตอบคำถามที่ถามซ้ำ ๆ หรือไม่จำเป็นต้องมานั่งเก็บข้อมูลทีละคน ช่วยให้ผู้ใช้งานแก้ไขปัญหาได้ในเบื้องต้นอย่างว่องไว ไม่ต้องรอคอยเป็นเวลานาน สร้างความประทับใจ ปิดการขายได้เร็วขึ้นและลดต้นทุนในการจ้างแอดมินเพื่อมาคอยตอบคำถามตลอดเวลา เพราะบริการนี้จะช่วยเหลือคุณได้ทุกอย่างที่สามารถทำได้ 
\subsubsection{การสร้าง LINE Bot โดยใช้ Dialogflow} 
\quad ประกอบด้วย 2 ส่วน คือ LINE Official Account เป็นส่วนที่เราต้องสร้างขึ้นเพื่อใช้ในการทำ LINE Bot ที่ไว้ใช้ในการโต้ตอบกับ Dialogflow สามารถกำหนดได้ทั้ง สติกเกอร์ รูปภาพ ข้อความ และ วีดีโอ และ Dialogflow เป็นแพลตฟอร์มที่สามารถช่วยในการพัฒนา LINE Bot ได้สามารถแบ่งได้ 2 กรณีดังนี้ 1.การเขียน Dialogflow ขั้นพื้นฐานไม่จำเป็นต้องทำการเขียนโปรแกรมเลย เนื่องจากเราสามารถพิมพ์ข้อความต่าง ๆที่ใช้สำหรับการถาม-ตอบได้เลย 2.การเขียน Dialogflow ขั้นสูงอาจจะมีการเขียนโปรแกรมเพื่อเพิ่มความสามารถของ LINE Bot ได้ เช่น การส่ง Location การส่งรูปภาพ การส่งสติกเกอร์ เป็นต้น 
\cite{lineBot} \cite{lineDf} \cite{lineDfPyFb}
\section{เทคโนโลยีที่เกี่ยวข้องอื่น ๆ}

\subsubsection{Visual Studio Code (VS Code)}
\quad Visual Studio Code หรือ VS Code เป็นโปรแกรม Code Editor ที่ใช้ในการแก้ไขและปรับแต่งโค้ด โดยมาจากค่ายไมโครซอฟท์ ที่มีการพัฒนาออกมาในรูปแบบของ Opensource จึงสามารถนำมาใช้งานได้แบบฟรี ๆ ทีสต้องการความเป็นมืออาชีพ ซึ่ง Visual Studio Code นั้น เหมาะสำหรับนักพัฒนาโปรแกรมที่ต้องการใช้งานกับแพลตฟอร์ม มีการรองรับการใช้งานทั้งบน Windows, macOS และ Linux มีการสนับสนุนทั้งภาษา  JavaScript, TypeScript และ Node.js สามารถเชื่อมต่อกับ  Git ได้ สามารถนำมาใช้งานได้ง่ายไม่ซับซ้อน มีเครื่องมือส่วนขยายต่าง ๆ ให้เราเลือกใช้อย่างมาก ไม่ว่าจะเป็น 1.การเปิดใช้งานภาษาอื่น ๆ ทั้ง ภาษา  C++,  C\#, Java, Python, PHP หรือ Go  2.Themes 3.Debugger 4.Commands 
\cite{vscode}
\subsubsection{Dialogflow} 
\quad Dialogflow คือ platform สำหรับสร้าง chatbot ของ Google ที่ใช้ machine learning ด้าน Natural Language Processing (NLP) มาช่วยในทำความเข้าใจถึงความต้องการ (intent) และสิ่งที่ต้องการ (entity) ในประโยคสนทนาของผู้ใช้งานและตอบคำถามตามความต้องการของผู้ใช้งาน ตามกฎ หรือ flow ที่ผู้พัฒนาวางเอาไว้ ซึ่ง Dialogflow จะช่วยเพิ่มความยืดหยุ่นของประโยคที่ chatbot รับมา ว่าไม่จำเป็นต้องตรงตามเงื่อนไข แบบ rule based เป๊ะ ๆ ก็สามารถเข้าใจถึงความต้องการของผู้ใช้งานได้ 
\cite{lineDf}
\subsubsection{Firebase} 
\quad Firebase คือ Platform ที่รวบรวมเครื่องมือต่าง ๆ สำหรับการจัดการในส่วนของ Backend(Server side) ซึ่งทำให้สามารถ Build Mobile Application ได้อย่างมีประสิทธิภาพ และยังลดเวลาและค่าใช้จ่ายของการทำ Server side หรือการวิเคราะห์ข้อมูลให้อีกด้วย โดยมีทั้งเครื่องมือที่ฟรี และเครื่องมีที่มีค่าใช้จ่าย (สำหรับการ Scale) 
\cite{firebase}
\subsubsection{Nuxt.js }
\quad Nuxt.js คือ Framework ที่นำ Vue.js มาสร้าง web application เสริมความสามารถในการทำ SSR และ Progressive Web Application (PWA) 
\cite{nuxt}

% \section{Third section}
% Section 3 text. The dielectric constant\index{dielectric constant}
% at the air-metal interface determines
% the resonance shift\index{resonance shift} as absorption or capture occurs
% is shown in Equation~\eqref{eq:dielectric}:

% \begin{equation}\label{eq:dielectric}
% k_1=\frac{\omega}{c({1/\varepsilon_m + 1/\varepsilon_i})^{1/2}}=k_2=\frac{\omega
% \sin(\theta)\varepsilon_\mathit{air}^{1/2}}{c}
% \end{equation}

% \noindent
% where $\omega$ is the frequency of the plasmon, $c$ is the speed of
% light, $\varepsilon_m$ is the dielectric constant of the metal,
% $\varepsilon_i$ is the dielectric constant of neighboring insulator,
% and $\varepsilon_\mathit{air}$ is the dielectric constant of air.

% \section{About using figures in your report}

% % define a command that produces some filler text, the lorem ipsum.
% \newcommand{\loremipsum}{
%   \textit{Lorem ipsum dolor sit amet, consectetur adipisicing elit, sed do
%   eiusmod tempor incididunt ut labore et dolore magna aliqua. Ut enim ad
%   minim veniam, quis nostrud exercitation ullamco laboris nisi ut
%   aliquip ex ea commodo consequat. Duis aute irure dolor in
%   reprehenderit in voluptate velit esse cillum dolore eu fugiat nulla
%   pariatur. Excepteur sint occaecat cupidatat non proident, sunt in
%   culpa qui officia deserunt mollit anim id est laborum.}\par}

% \begin{figure}
%   \centering

%   \fbox{
%      \parbox{.6\textwidth}{\loremipsum}
%   }

%   % To include an image in the figure, say myimage.pdf, you could use
%   % the following code. Look up the documentation for the package
%   % graphicx for more information.
%   % \includegraphics[width=\textwidth]{myimage}

%   \caption[Sample figure]{This figure is a sample containing \gls{lorem ipsum},
%   showing you how you can include figures and glossary in your report.
%   You can specify a shorter caption that will appear in the List of Figures.}
%   \label{fig:sample-figure}
% \end{figure}

% \loremipsum\loremipsum

% % This code demonstrates how to get a landscape table or figure. It
% % uses the package lscape to turn everything but the page number into
% % landscape orientation. Everything should be included within an
% % \afterpage{ .... } to avoid causing a page break too early.
% \afterpage{
%   \begin{landscape}
%   \begin{table}
%     \caption{Sample landscape table}
%     \label{tab:sample-table}

%     \centering

%     \begin{tabular}{c||c|c}
%         Year & A & B \\
%         \hline\hline
%         1989 & 12 & 23 \\
%         1990 & 4 & 9 \\
%         1991 & 3 & 6 \\
%     \end{tabular}
%   \end{table}
%   \end{landscape}
% }

% \loremipsum\loremipsum\loremipsum

% \section{Overfull hbox}

% When the \verb.semifinal. option is passed to the \verb.cpecmu. document class,
% any line that is longer than the line width, i.e., an overfull hbox, will be
% highlighted with a black solid rule:
% \begin{center}
% \begin{minipage}{2em}
% juxtaposition
% \end{minipage}
% \end{center}

% \section{\ifcpe%
% ความรู้ตามหลักสูตรซึ่งถูกนำมาใช้หรือบูรณาการในโครงงาน
% \else%
% ISNE knowledge used, applied, or integrated in this project
% \fi
% }

% อธิบายถึงความรู้ และแนวทางการนำความรู้ต่างๆ ที่ได้เรียนตามหลักสูตร ซึ่งถูกนำมาใช้ในโครงงาน

% \section{\ifcpe%
% ความรู้นอกหลักสูตรซึ่งถูกนำมาใช้หรือบูรณาการในโครงงาน
% \else%
% Extracurricular knowledge used, applied, or integrated in this project
% \fi
% }

% อธิบายถึงความรู้ต่างๆ ที่เรียนรู้ด้วยตนเอง และแนวทางการนำความรู้เหล่านั้นมาใช้ในโครงงาน
